\chapter{Physical Input Devices}
\label{Chapter:PhysicalInputDevices}




\section{Overview}
This chapter deals with the actual, physical devices an, for instance human being, is using in the physical world in order to communicate with the computer. While starting with traditional input devices like the keyboard or the mouse, we also cover classic gaming devices like the joystick/gamepad and modern gaming devices like Wii Remote\footnote{Sometimes unofficially nicknamed \emph{Wiimote}}. Especially for developers, there are input devices like a space mouse making navigation within virtual worlds quite comfortable - therefore, they are covered as well.




\section{Keyboard}
As seen within source~code~\ref{Code:KeyboardUsageExample}, accessing a keyboard is quite simple.
\begin{lstlisting}[float=htb,label=Code:KeyboardUsageExample,caption={Keyboard usage example}]
// Get keyboard input device
Keyboard *pKeyboard = InputManager::GetInstance()->GetKeyboard();
if (pKeyboard) {
  // Update movement vector
  if (pKeyboard->KeyUp.IsPressed())
    vMovement += vDirVector;
  if (pKeyboard->KeyDown.IsPressed())
    vMovement -= vDirVector;
}
\end{lstlisting}




\section{Mouse}
As seen within source~code~\ref{Code:MouseUsageExample}, accessing a mouse is quite simple.
\begin{lstlisting}[float=htb,label=Code:MouseUsageExample,caption={Mouse usage example}]
// Get mouse input device
Mouse *pMouse = InputManager::GetInstance()->GetMouse();
if (pMouse && pMouse->Left.IsPressed()) {
  // Get the current time difference
  const float fTimeDiff = Timing::GetInstance()->GetTimeDifference();

  // Get mouse movement
  const float fX = pMouse->X.GetValue() * fTimeDiff;
  const float fY = pMouse->Y.GetValue() * fTimeDiff;

  // Get a quaternion representation of the rotation delta
  Quaternion qRotInc;
  EulerAngles::ToQuaternion(float(fX * Math::DegToRad),
                            float(fY * Math::DegToRad),
                            0.0f,
                            qRotInc);

  // Set new rotation
  qNewRotation = qOldRotation * qRotInc;
}
\end{lstlisting}




\section{Joystick}
When talking about \emph{joysticks}, we also cover \emph{gamepads} by this term because technically, their're practically the same. Therefore, when looking over the PixelLight input device classes, one may suspect that there's no support for gamepads, but that's not true because joysticks and gamepads are both handled by the joystick class.

As seen within source~code~\ref{Code:JoystickUsageExample}, accessing a joystick is quite simple.
\begin{lstlisting}[float=htb,label=Code:JoystickUsageExample,caption={Joystick usage example}]
Joystick *pJoystick = (Joystick*)InputManager::GetInstance()->GetDevice("Joystick0");
if (pJoystick && pJoystick->GetButtons()[0]->IsPressed()) {
  // Get the current time difference
  const float fTimeDiff = Timing::GetInstance()->GetTimeDifference();

  // Get joystick axis
  const float fX = pJoystick->X.GetValue() * fTimeDiff;
  const float fY = pJoystick->Y.GetValue() * fTimeDiff;

  // Get a quaternion representation of the rotation delta
  Quaternion qRotInc;
  EulerAngles::ToQuaternion(float(fX * Math::DegToRad),
                            float(fY * Math::DegToRad),
                            0.0f,
                            qRotInc);

  // Set new rotation
  qNewRotation = qOldRotation * qRotInc;
}
\end{lstlisting}




\section{Wii Remote}
As seen within source~code~\ref{Code:WiiRemoteUsageExample}, accessing a Wii Remote is quite simple.
\begin{lstlisting}[float=htb,label=Code:WiiRemoteUsageExample,caption={Wii Remote usage example}]
// Get Wii Remote input device
WiiMote *pWiiMote = (WiiMote*)InputManager::GetInstance()->GetDevice("WiiMote0");
if (pWiiMote && (pWiiMote->ButtonA.IsPressed())) {
  // Get the current time difference
  const float fTimeDiff = Timing::GetInstance()->GetTimeDifference();

  // Get orientation
  const float fX = float(pWiiMote->OrientX.GetValue() * fTimeDiff * Math::DegToRad);
  const float fY = float(pWiiMote->OrientY.GetValue() * fTimeDiff * Math::DegToRad);
  const float fZ = float(pWiiMote->OrientZ.GetValue() * fTimeDiff * Math::DegToRad);

  // Get a quaternion representation of the rotation delta
  Quaternion qRotInc;
  EulerAngles::ToQuaternion(fX, fY, fZ, qRotInc);

  // Set new rotation
  qNewRotation = qOldRotation * qRotInc;
}
\end{lstlisting}




\section{Space mouse}
Several space mouses from \emph{3DConnexion} are supported: \emph{SpaceMousePlus}, \emph{SpaceTraveler}, \emph{SpaceBall}, \emph{SpacePilot} and \emph{SpaceExplorer}.

Controlling an orbiting or free camera by using a space mouse is quite comfortable and is replacing mouse AND keyboard for this purpose.

As seen within source~code~\ref{Code:SpaceMouseUsageExample}, accessing a space mouse is quite simple.
\begin{lstlisting}[float=htb,label=Code:SpaceMouseUsageExample,caption={Space mouse usage example}]
// Get SpaceMouse device
InputManager *pInputManager = InputManager::GetInstance();
SpaceMouse *pSpaceMouse = (SpaceMouse*)pInputManager->GetDevice("SpaceMouse0");
if (pSpaceMouse) {
  // Get the current time difference
  const float fTimeDiff = Timing::GetInstance()->GetTimeDifference();

  // Get orientation
  const float fX = float(pSpaceMouse->RotX.GetValue() * fTimeDiff * Math::DegToRad);
  const float fY = float(pSpaceMouse->RotY.GetValue() * fTimeDiff * Math::DegToRad);
  const float fZ = float(pSpaceMouse->RotZ.GetValue() * fTimeDiff * Math::DegToRad);

  // Get a quaternion representation of the rotation delta
  Quaternion qRotInc;
  EulerAngles::ToQuaternion(fX, fY, fZ, qRotInc);

  // Set new rotation
  qNewRotation = qOldRotation * qRotInc;

  // Get translation
  const Vector3 vTrans(pSpaceMouse->TransX.GetValue() * fTimeDiff, pSpaceMouse->TransY.GetValue() * fTimeDiff, pSpaceMouse->TransZ.GetValue() * fTimeDiff);

  // Set new position
  vNewPos = vOldPos + vTrans;
}
\end{lstlisting}
