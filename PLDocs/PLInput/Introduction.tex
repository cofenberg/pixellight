\chapter{Introduction}


\paragraph{Motivation}
The PixelLight input component provides access and control over several input devices. Although the purpose of this project is to have an input library that perfectly integrates into the PixelLight framework, \emph{PLInput} can be used without other PixelLight components like the rendering system as well.




\section{External Dependences}
The core of PLInput depends on the \textbf{PLGeneral}, \textbf{PLCore} and \textbf{PLMath} libraries. PLInput is using some platform dependent third party libraries, but usually, the resulting binary of PLInput library is stand alone and does not force you to deliver additional shared libraries, too.


\paragraph{Microsoft Windows}
When compiling for \emph{Microsoft Windows}, you'll need the the external headers \emph{hidpi.h}, \emph{hidsdi.h} and \emph{hidusage.h} and the following libraries:
\begin{itemize}
\item \emph{winmm.lib} linking to \emph{winmm.dll}
\item \emph{hid.lib} linking to \emph{hid.dll}
\item \emph{setupapi.lib} linking to \emph{setupapi.dll}
\end{itemize}
Usually, the \emph{dll} files should be already available on the systems of end users. When using \emph{Microsoft Visual Studio}, some of the required libraries and headers may already be there. In general, this libraries can be found within the \emph{Windows Driver Kit (WDK)}\footnote{ISO (\SI{620}{\mebi\byte}) which can be freely downloaded from \url{http://www.microsoft.com/whdc/Devtools/wdk/default.mspx}}. To avoid legal issues, we don't include this files you'll need to compile the PLInput library. The official compiled PixelLight SDK is using the files from an older WDK version, previously known as \emph{Windows Driver Development Kit (DDK)}. We noticed that in newer WDK versions like \emph{7.1.0} some changes were made and the mentioned files are no longer enough.
