\chapter{Renderer backends}
This chapter deals with the different renderer backends shipping with the PixelLight SDK.




\section{Null}
\begin{itemize}
\item The null renderer backend is for situations were you don't need \emph{real} rendering output. It consists of dummy functions which will try to \emph{emulate} the set \& get functions that good as possible.
\item PixelLight component name: PLRendererNull
\item Used dll's: \emph{PLRendererNull.dll}
\end{itemize}




\section{OpenGL}
\begin{itemize}
\item This is the preferred renderer backend of the PixelLight team because OpenGL\footnote{More information about OpenGL can be found at \url{http://www.opengl.org/}} provides some nice features like line width and it's also supported by other operation systems like Linux.
\item PixelLight component name: PLRendererOpenGL
\item Used dll's: \emph{PLRendererOpenGL.dll}
\item FreeType\footnote{FreeType can be downloaded from \url{http://www.freetype.org/}} is used for font support.
\end{itemize}

Here's a list of some general notes:
\begin{itemize}
\item Normally when using rectangle textures (none power of 2) under OpenGL, you have to work with none normalized texture coordinates - but because we don't want to deal with texture coordinates for rectangle textures in a special way, within the backend the texture matrix is scaled. So, you always work with normalized texture coordinates. But if you use rectangle textures within a shader, you should take the texture matrix within your vertex shader into account!
\end{itemize}

There's a plugin for PLRendererOpenGL adding support for the Cg\footnote{(\url{http://developer.nvidia.com/object/cg_toolkit.html})} shader language:
\begin{itemize}
\item PixelLight component name: PLRendererOpenGLCg
\item Used dll's: \emph{PLRendererOpenGLCg.dll}, \emph{cg.dll} and \emph{cgGL.dll}
\end{itemize}




\section{OpenGL ES 2.0}
\begin{itemize}
\item Currently not within the official PixelLight SDK
\item OpenGL ES 2.0 renderer backend for mobile devices
\item PixelLight component name: PLRendererOpenGLES
\item FreeType\footnote{FreeType can be downloaded from \url{http://www.freetype.org/}} used for font support.
\end{itemize}




\section{Direct3D 9}
\begin{itemize}
\item Currently not within the official PixelLight SDK
\item This Direct3D 9\footnote{For more Direct3D information, please have a look at \url{http://www.microsoft.com/downloads/Browse.aspx?displaylang=de&categoryid=2}} renderer backend is for \emph{Microsoft Windows} systems only, and has some missing features.
\item PixelLight component name: PLRendererD3D9
\item Used dll's: \emph{PLRendererD3D9.dll}, \emph{cg.dll} and \emph{cgD3D9.dll} and DirectX 9 must be installed
\end{itemize}

Here's a list of some general notes:
\begin{itemize}
\item No line width support. \emph{SetLineWidth()} has no effect and \emph{GetLineWidth()} will always return 1
\item Because D3D doesn't have any constant color, \emph{SetColor()} must be emulated within this backend using for instance a vertex shader
\end{itemize}
Further, within this backend there are still some bugs to fix.
