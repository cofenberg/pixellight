%----- Section: Framework ------------------------
\section{Framework}


%----- SubSection: Creating a new document -------
\subsection{Creating a new document}
For creating a new document, just copy the files in /skeleton to a new directory and adjust
the file names (also change 'make.bat').


%----- SubSection: Using make.bat ----------------
\subsection{Using make.bat}
This batch file can be used to automatically run LaTeX for creating the document. It can be used
with the following options:\\

\begin{tabular}{ll}
\textbf{make}      & creates a .dvi file\\
\textbf{make pdf}  & creates a .pdf file\\
\textbf{make html} & creates .html file\\
\textbf{make all}  & creates .pdf and .html files\\
\end{tabular}


%----- SubSection: Adjusting the styles ----------
\subsection{Adjusting the styles}
You can adjust the used styles by editing 'styles.sty'. It contains of several options that are
used by the new commands:\\

\begin{tabular}{ll}
\textbf{link}       & Color of links \\
\textbf{NOTEframe}  & Frame color of the NOTE environment \\
\textbf{NOTEback}   & Background color of the NOTE environment \\
\textbf{NOTEprefix} & Prefix printed in the NOTE environment \\
\textbf{CODEframe}  & Frame color of the CODE environment \\
\textbf{CODEback}   & Background color of the CODE environment \\
\textbf{CODEprefix} &Prefix printed in the CODE environment \\
\end{tabular}
