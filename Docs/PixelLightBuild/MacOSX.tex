\chapter{Mac OS X}
This chapter explains how to compile PixelLight for \emph{Mac OS X 10.6}. Because not everyone is familiar with the Apple world, but might want to just compile PixelLight in order to have a first look into it, this chapter also explains some useful Mac OS X basics \footnote{Mac OS X is definitely made for end users, not developers. Stuff which is common on other operation systems, often is quite good hidden from the user in the Mac world.}.


\paragraph{Under Construction}
The Mac OS X port is currently under construction.


\paragraph{Compiler Version}
When porting PixelLight to Mac OS X in 2011, there were several issues with the \ac{GCC} 4.2.1 used on the available system. During the port process, Xcode 4.2 was released introducing C++11 support. At the time of writing this document this new version was not yet successfully installed on the used \emph{Mac OS X 10.6} system. PixelLight also compiles without C++11 features, but you might want to use a more up-to-date compiler like \ac{GCC} 4.6. Have a look at appendix~\ref{Appendix:MacOSX_GCC} for more details about this topic.


\paragraph{Linux}
On the terminal level, Mac OS X is nearly identical to Linux. So, there's no point in copy'n'paste the Linux part. Instead, this section will refer to the relevant Linux sections.


\paragraph{External Packages}
The management of external packages is the same on all platforms. Refer to to the Linux section~\ref{Chapter:Linux_ExternalPackages} or directly to chapter~\ref{Chapter:ExternalDependencies}. On Mac OS X and \SI{32}{\bit}, put everything in the directory \emph{External/\_MacOSX\_x86\_32}.


\paragraph{CMake}
CMake as described in \ref{Chapter:Linux_CMake} can be used in the same way under Mac OS X.


\paragraph{Maketool}
The maketool script as described in \ref{Chapter:Linux_Maketool} can be used in the same way under Mac OS X.


\paragraph{Build and Run}
You may also want to have a look at the Linux sections \ref{Chapter:Linux_Build} and \ref{Chapter:Linux_RunningFromALocalBuildAndInstalling}.


\paragraph{1. Run from a your Local Source Directory}
Within your home (\emph{\textasciitilde}) directory, open the hidden \emph{\textasciitilde /.bash\_profile} file and follow the Linux section \ref{Chapter:Linux_Build} explaining how to run from a your local source directory.



\subsection{Mac Newbie}
Mac geeks can skip this section, all other might find it useful to make a first step into the Apple world while still being able to at least keep some of the hairs.


\paragraph{Show Hidden Files in Finder}
\emph{Finder}, the standard file browser in the Apple world, by default does not show hidden files and offers no \ac{GUI}-option to change that. As developer, you might want to also see and edit hidden files, for example within the hidden Git-repository directory or for files like \emph{\textasciitilde /.bash\_profile}.

So, open a terminal and type
\begin{lstlisting}[language=sh]
defaults write com.apple.finder AppleShowAllFiles TRUE
killall Finder
\end{lstlisting}
The first line sets the \emph{Finder} option we're interested in, the second kills all currently opened \emph{Finder} instances. The changed option has only an effect for new \emph{Finder} instances.
 

\paragraph{Keyboard}
It's useful to know which key combination to press to type totally fundamental characters which are usually on decent PC keyboards, but not the standard Apple ones.

\begin{itemize}
	\item \textasciitilde key, you may need for \emph{\textasciitilde /.bash\_profile}: Press alt-n
	\item \textbackslash key, alt-shift-7
	\item $[$ key, alt-5
	\item $]$ key, alt-6
	\item \textbar key, alt-7
	\item \textbraceleft key, alt-8
	\item \textbraceright key, alt-9
\end{itemize}


\paragraph{ObjC}
Unlike MS Windows, Linux or Android, one has to use ObjC instead of C/C++ in order to be able to use the Mac OS X API's. It's possible to use C/C++ within ObjC, but not the other way around. In order to combine C/C++ and ObjC, it's therefore important to keep the headers clean and to only use C/C++ within headers which will included within C/C++ codes.

In case you are unfamiliar with ObjC and don't want to learn another programming language just in order to be able to access the Mac OS X API, you might want to read the document \emph{From C++ to Objective-C}\footnote{\url{http://pierre.chachatelier.fr/programmation/fichiers/cpp-objc-en.pdf}} from Pierre Chatelier. This document provides an excellent comparison between C++ and ObjC which can be used to learn the syntax differences on the fly while working with the unfamiliar programming language.
