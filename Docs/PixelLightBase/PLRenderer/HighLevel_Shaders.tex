%----- Chapter: High level: Shaders --------------
\chapter{High level: Shaders}


[TODO] Update!



%----- Section: Overview -------------------------
\section{Overview}
In general PixelLight is using the Cg shader language for vertex and fragment shader effects. Have 
a look at e.g. \url{http://www.shadertech.com/} or developer.nvidia.com/cg to learn the basics
about this technology. Because Cg is \ac{API} and platform independent you can work with Cg functions
directly - therefore have a look at the Cg documentation to see how to use the Cg functions.\\
The PixelLight material is encapsulating this shaders and you haven't to work with them directly. But
because you can do alot with shaders you have also low level access to the shaders to be able
to create e.g. grass effects etc.\\

Vertex/fragment programs will process every vertices/fragments on the \ac{GPU}. When using such programs
you normally will only work with the provided program parameters. The PixelLight shader system will give the
programmer the ability to use this shaders in the application.\\
The PixelLight shader handler PLTShaderHandler provides an interface with all required functions to work with
such programs. Each individual program needs it's own derived shader handler because each program has
different parameters and functionality! So, if you want to use a certain vertex/fragment program you
have to use it through its own shader handler. The shader handler is the obverse of the program on the \ac{GPU}.




%----- Section: Shader handler -------------------
\section{Shader handler}
The main shader handler functions are Load(), Bind() and Update(). First you have to load up such
a vertex/fragment program using the Load() function of the shader handler. If the program filename
ends with a '+' it's a fragment program, else a vertex program. The filename ending of the programs
is '.cg'.\\
Then you should update the shader hander each frame to ensure that the internal parameters are
updated correctly. Before rendering anything the shader is activated using the Bind() function and
after the render process is finished it should be deactivated using Unbind().\\
Note that only one vertex and one fragment program can be active at the same time!\\
In the example below two derived shader handlers are used:\\

\begin{lstlisting}[caption=Using shader handler]
// Somewere in e.g. your class
TVertexShaderHandler   m_cVertexShader;
TFragmentShaderHandler m_cFragmentShader;

// Load somewere the vertex and fragment programs
  m_cVertexShader.Load("myVertexShader");
m_cFragmentShader.Load("myFragmentShader+");

// Render process
// Bind shaders
m_cVertexShader.Bind();
m_cFragmentShader.Bind();

// Draw anything

// Unbind shaders
m_cFragmentShader.Unbind();
m_cVertexShader.Unbind();

// Somewere in your code...
// Update shader handlers
m_cFragmentShader.Update();
m_cVertexShader.Update();
\end{lstlisting}




%----- Section: Derive shader handler ------------
\section{Derive shader handler}
To be able to use your own program you have to derive PLTShaderHandler to implement your shader
handler which will manage this the program.\\
After a program was loaded the virtual function CustomLoad() is called. There you should initialize
your handler and getting the program parameters. If the shader handler is bind the function CustomBind()
is called were the program parameters should be set. Finally use CustomUpdate() to update your shader
handler. Example:\\

\begin{lstlisting}[caption=Creating own shader handler]
class TVertexShaderHandler : public PLTShaderHandler {

	// Private data
	private:
		float m_fTime;

	// Programm parameters
	private:
		CGparameter m_ConstantsParam;
		CGparameter m_ModelViewProjParam;
		CGparameter m_ModelViewITParam;
		CGparameter m_ModelViewParam;

	// Virtual PLTShaderHandler functions
	private:
		virtual void CustomLoad();
		virtual void CustomBind();
		virtual void CustomUpdate();


};

void TVertexShaderHandler::CustomLoad()
{
    // Init data
    m_fTime = 0.f;

    // Get program variables
    m_ConstantsParam     = GetCGNamedParameter("Constants");
    m_ModelViewProjParam = GetCGNamedParameter("ModelViewProj");
    m_ModelViewITParam   = GetCGNamedParameter("ModelViewIT");
    m_ModelViewParam     = GetCGNamedParameter("ModelView");
}

void TShaderGrass::CustomBind()
{
    // Set program parameters
             cgGLSetParameter4f(m_ConstantsParam,     m_fTime, 0, 0, 1);
    cgGLSetStateMatrixParameter(m_ModelViewProjParam, CG_GL_MODELVIEW_PROJECTION_MATRIX, CG_GL_MATRIX_IDENTITY);
    cgGLSetStateMatrixParameter(m_ModelViewITParam,   CG_GL_MODELVIEW_MATRIX,            CG_GL_MATRIX_INVERSE_TRANSPOSE);
    cgGLSetStateMatrixParameter(m_ModelViewParam,     CG_GL_MODELVIEW_MATRIX,            CG_GL_MATRIX_IDENTITY);
}

void TShaderGrass::CustomUpdate()
{
    // Update timer
    m_fTime += PL::Timer.GetTimeDifference();
}
\end{lstlisting}




%----- Section: Material shader effects ----------
\section{Material shader effects}
In the shader section above you saw how to use shader effects as programmer directly in your program
by programming their usage by hand for special situations like grass effects were you also have to
create and setup the vertices in the correct way.
Because the vertex/fragment programs are extreme implementation dependent and the programer has to
do the most work graphic artists haven't that many control over them and new shader effects can't be
used without write some new code in your application - except the new effect is using the same parameters
etc. as another effect.\\
Therefore PixelLight has an advanced effect system which is implemented in the material itself to give the
graphic artists more control over the appearance of  surfaces etc - but in reverse the programmer
itself has less control because he isn't able to manipulate this cg programs directly in his code.\\




%----- Section: CgFX -----------------------------
\section{CgFX}
During development graphic artists can use CgFX which also provides the environment the vertex/fragment
programs will work in and is setting its own render states, used textures etc. Further the fx files
provide in contrast to cg files also a description of the program parameters with default values, \ac{GUI}
definitions in annotations to manipulare this values in editors using \ac{GUI} elements etc.
Also different techniques/fallbacks can be defined. The 'fx' file format is nealy the same as the
DirectX effect format and if you write your own programs in there it is identical to Cg or assembly
programs. There are plugins for \emph{Autodesk 3ds Max} etc which enables the graphic artists to develop the
advanced shader effects directly in their well kown development environment! The CgFX 'fx' file format
can be compared with the PixelLight own material file format 'mat', therefore is isn't that difficult to 
convert such CgFX effects artists had developed in their own development environment into the
PixelLight own material format.\\

PixelLight itself doesn't supply direct CgFX support because it doesn't provide a nice programmer
interface and is setting render states etc. automatically without enabling the programmer to find
out what in fact was set. Further CgFX isn't THAT compatible to other graphic cards (e.g. ATI) like
Cg itself which would result in many troubles. Therefore we decided to don't implementet CgFX directly
- but as mentioned above, the PixelLight own material format has nearly the same capability as CgFX
without the compatibility troubles CgFX has!\\

The material itself which can contain shaders is nearly used in the same way as the 'by hand' Cg shaders
are used. But unlike the direct shader interface the material effects provides a more comfortable
parameter interface. Further there are different known parameter semantics which will setup the
shader parameter automatically (e.g. set current projection matrix). Therefore you mustn't derive
the PLTMaterial to deal with the different shader parameters. Some parameters with kown semantic
will be managed complete automatically and the other you can manipulate in a quite comfortable way
over the interface of the material.\\




%----- Section: Material shader parameters -------
\section{Material shader parameters}
The material shader parameters are split up into two parameter types:\\
- Parameter: Parameters were the semantic is unknown.
             They can be manipulated by hand.\\
- Semantic:  Parameters were the semantic is known.
             They will be updated automatically and shouldn't be manipulated
             by hand.\\




%----- Section: Parameter semantic ---------------
\section{Parameter semantic}
Here's a list of all known parameter semantic:\\

 \begin{tabular}{|p{7cm}|p{7cm}|}
\hline
\textbf{Parameter semantic} & \textbf{Description}\\
\hline
World               & World matrix (float4x4)\\
\hline
WorldI	            & World inverse matrix (float4x4)\\
\hline
WorldIT	            & World inverse transpose matrix (float4x4)\\
\hline
WorldView           & World view matrix (float4x4)\\
\hline
WorldViewI          & World view inverse matrix (float4x4)\\
\hline
WorldViewIT         & World view inverse transpose matrix (float4x4)\\
\hline
WorldViewProjection & World view projection matrix (float4x4)\\
\hline
Projection	        & Projection matrix (float4x4)\\
\hline
View	            & View matrix (float4x4)\\
\hline
ViewIT	            & View inverse transpose matrix (float4x4)\\
\hline
EyePos	            & Eye position (object space, float4)\\
\hline
EyeVector           & Eye direction vector (object space, float3)\\
\hline
EyePosWorld	        & Eye position (world space, float4)\\
\hline
EyeVectorWorld      & Eye direction vector (world space, float3)\\
\hline
Time	            & Timer (float)\\
\hline
Shininess	        & Shininess (float)\\
\hline
Color	            & Color (float4)\\
\hline
AmbientColor        & Ambient color (float4)\\
\hline
DiffuseColor        & Diffuse color (float4)\\
\hline
SpecularColor       & Specular color (float4)\\
\hline
EmissionColor       & Emission color (float4)\\
\hline
Texture<n>          & Texture 0-7 (texture)\\
\hline
TextureMatrix<n>    & Texture matrix 0-7\\
\hline
\end{tabular}
