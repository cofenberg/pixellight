\section{System}




\subsection{OS console}
Within the \emph{Console}-class you can find some basis OS console functions. Those can be quite handy because even the most primitive functions can and often are different on multiple platforms! You can request an instance of this class by using the \emph{System}-class.

The following example shows how you can use some of this functions for a simple but useless application:

\begin{lstlisting}[caption=OS console usage example]
//[-------------------------------------------------------]
//[ Includes                                              ]
//[-------------------------------------------------------]
#include <PLCore/Main.h>
#include <PLCore/System/System.h>
#include <PLCore/System/Console.h>


//[-------------------------------------------------------]
//[ Namespaces                                            ]
//[-------------------------------------------------------]
using namespace PLCore;


//[-------------------------------------------------------]
//[ Program entry point                                   ]
//[-------------------------------------------------------]
int PLMain(const String &sFilename, const Array<String> &lstArguments)
{
  // Get the console
  const Console &cConsole = System::GetInstance()->GetConsole();

  // This code loops until any key is hit
  cConsole.Print("\nPress any key to quit...\n");
  while (!cConsole.IsKeyHit()) {
    // Do anything
  }

  // Read and throw away the hit key
  cConsole.GetCharacter();

  // Clear the OS console
  cConsole.ClearScreen();

  // Ask the user something
  cConsole.Print("\nDo you want a sweet goodbye? (y/n)\n");
  const int nCharacter = cConsole.GetCharacter();
  if (nCharacter == 'y' || nCharacter == 'Y') {
    cConsole.Print("Goodbye\n");

    // Wait 2 seconds
    System::GetInstance()->Sleep(2000);
  }

  // Done
  return 0;
}
\end{lstlisting}
