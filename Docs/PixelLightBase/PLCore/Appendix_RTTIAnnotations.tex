\chapter{PLCore - RTTI Annotations}
\label{Appendix:RTTIAnnotations}
RTTI objects like attributes have a string for optional annotations. In here, additional information or hints about the RTTI objects can be provided. Within the RTTI, the annotation has no build in meaning. Use annotations to denote for example control information for a GUI system, like a minimum/maximum value or whether or not a string is used as a filename, so a GUI can offer the user a file dialog.

Table~\ref{Table:PixelLightAnnotations} shows annotation semantics used across PixelLight itself:
\begin{table}[htb]
	\centering
	\begin{tabular}{|l|p{0.88\textwidth}|}
		\hline
		\textbf{Name} & \textbf{Semantic}\\
		\hline
		\hline
		Min  & Minimum allowed value. Example: \emph{Min=`0 1 2`} or \emph{Min=`-1`}\\
		\hline
		Max  & Maximum allowed value. Example: \emph{Max=`0 1 2`} or \emph{Max=`1`}\\
		\hline
		Inc  & For instance for spinner increase/decrease speed, default value is $0.1$. Example: \emph{Inc=5} or \emph{Inc=`0.2`}\\
		\hline
		Ext  & Valid filename extensions. Example: \emph{Ext=`mat jpg bmp`}\\
		\hline
		Type & Denotes the type of a resource that can be used. Example \emph{Type='Image'} to denote that it's a loadable resource of the type \emph{Image} or \emph{Type='Material Effect Image TextureAni'}.\\
		\hline
	\end{tabular} 
	\caption{PixelLight annotations for \emph{<name>=<value>}}
	\label{Table:PixelLightAnnotations}
\end{table}
