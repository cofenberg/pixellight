\section{Physics Backends}
This section deals with the different physics backends shipping with the PixelLight SDK.


\paragraph{Null}
\begin{itemize}
\item The null physics backend does nothing - can be useful if you e.g. are in debug mode and don't want have any physics.
\item PixelLight component name: PLPhysicsNull
\item Used dll's: \emph{PLPhysicsNull.dll}
\end{itemize}


\paragraph{Newton Game Dynamics}
\begin{itemize}
\item This is the preferred physics plugin of the PixelLight team because Newton Game Dynamics\footnote{Newton Game Dynamics can be downloaded from \url{http://newtondynamics.com/}} is really easy to use, offers a lot of geometry and joint types by default, has a nice interface (in fact JUST one header) and some usefull tool functions. Also the community is very active.
\item PixelLight component name: PLPhysicsNewton
\item Used dll's: \emph{PLPhysicsNewton.dll} and \emph{newton.dll}
\end{itemize}


\paragraph{Open Dynamics Engine (ODE)}
\begin{itemize}
\item Currently not within the official PixelLight SDK
\item The Open Dynamics Engine\footnote{Open Dynamics Engine can be downloaded from \url{http://www.ode.org/}} or short ODE is a well-known open source physics engine.
\item PixelLight component name: PLPhysicsODE
\item Used dll's: \emph{PLPhysicsODE.dll} and \emph{ode.dll}
\end{itemize}


\paragraph{PhysX}
\begin{itemize}
\item Currently not within the official PixelLight SDK
\item PhysX\footnote{PhysX can be downloaded from \url{http://developer.nvidia.com/object/physx.html}} is a commercial multithreaded physics API with support for the physics GPU. Please keep in mind that the \emph{PhysX SDK System Software} must be installed to be able to use PhysX - this may not be acceptable for each project.
\item PixelLight component name: PLPhysicsPhysX
\item Used dll's: \emph{PLPhysicsPhysX.dll} and \emph{PhysXLoader.dll}
\end{itemize}
