\chapter{Scene Examples}
\label{Appendix:SceneExamples}
This appendix contains several examples and code snipes regarding to scenes. By using the shown features, one can
\begin{itemize}
\item{Construct a scene completely by using a script}
\item{Load in a complete scene at once by using a script}
\item{Use a hybrid approach, construct parts by using a script while other sub-scenes are loaded in}
\end{itemize}
In this appendix, the global variable \emph{this} points to the C++ \ac{RTTI} application class instance invoking the script 




\section{Constructing Scenes}
\begin{lstlisting}[language=lua]
function OnInit()
	this:ClearScene()
	local sceneContainer = this:GetScene()
	if sceneContainer ~= nil then
		sceneContainer:Create("PLScene::SNMesh", "Floor", "Position='0 -2.1 5' Scale='4 0.1 4' Rotation='0 180 0' Mesh='Default'")
		sceneContainer:Create("PLScene::SNMesh", "Box", "Position='0 -1.5 5' Scale='0.5 0.5 0.5' Mesh='Default'")
		sceneContainer:Create("PLScene::SNDirectionalLight", "Sun", "Rotation='45 0 0'")
		local camera = sceneContainer:Create("PLScene::SNCamera", "FreeCamera")
		if camera ~= nil then
			camera:AddModifier("PLEngine::SNMEgoLookController")
			camera:AddModifier("PLEngine::SNMMoveController")
		end
		this:SetCamera(camera)
	end
end
\end{lstlisting}

Instead of
\begin{lstlisting}[language=lua]
sceneContainer:Create("PLScene::SNMesh", "Floor", "Position='0 -2.1 5' Scale='4 0.1 4' Rotation='0 180 0' Mesh='Default'")
\end{lstlisting}
one could also write
\begin{lstlisting}[language=lua]
local sceneNode = sceneContainer:Create("PLScene::SNMesh")
if sceneNode ~= nil then
	sceneNode.Name = "Floor"
	sceneNode.Position = "0 -2.1 5"
	sceneNode.Scale = "4 0.1 4"
	sceneNode.Mesh = "Default"
end
\end{lstlisting}
... but the first version is more handy in this situation.

Have a look at \emph{45ScriptApplication.lua} within the PixelLight \ac{SDK} for a more advanced example on how to construct complete scenes by using a script.




\section{Loading Scenes}
In this section we're going to use the plugin \emph{PLAssimp} using \ac{ASSIMP} to be able to load in a complete scene, without merging everything into one huge mesh. The following example is using the x-scene from \url{http://www.daniel-sadowski.com/nyx/}.

Download \url{http://www.daniel-sadowski.com/nyx/nyx_exec.exe} and extract it. Then create a Lua script file within the same directory as "Nyx.exe" with the following content:
\begin{lstlisting}[language=lua]
function OnInit()
	this:SetBaseDirectory("data/tex/")	-- The loaded scene is using just texture names like "Houses"
	this:ClearScene()
	local sceneContainer = this:GetScene()
	if sceneContainer ~= nil then
		sceneContainer:Create("PLScene::SceneContainer", "Serenael", "Scale='0.05 0.05 0.05' Filename='data/msh/serenael.x'")
		sceneContainer:Create("PLScene::SNDirectionalLight", "Sun", "Rotation='45 0 0'")
		local camera = sceneContainer:Create("PLScene::SNCamera", "FreeCamera")
		if camera ~= nil then
			camera:AddModifier("PLEngine::SNMEgoLookController")
			camera:AddModifier("PLEngine::SNMMoveController")
		end
		this:SetCamera(camera)
	end
end
\end{lstlisting}
... and just load this script with PLViewer and you should be able to fly through this fantasy city within the PixelLight viewer.

You can also replace
\begin{lstlisting}[language=lua]
sceneContainer:Create("PLScene::SceneContainer", "Serenael", "Scale='0.05 0.05 0.05' Filename='data/msh/serenael.x'")
\end{lstlisting}
by
\begin{lstlisting}[language=lua]
sceneContainer:Create("PLScene::SNMesh", "Serenael", "Scale='0.05 0.05 0.05' Mesh='data/msh/serenael.x'")
\end{lstlisting}
in order to load in this x-scene as one single huge mesh. By the way, \ac{ASSIMP} was able to detect 80 mesh instances in this scene, meaning it optimizes the data quite well.

\emph{PLScene::SceneContainer} offers a RTTI method named \emph{LoadByFilename()}. With this method, one can now also write within a Lua script e.g.
\begin{lstlisting}[language=lua]
local serenaelContainer = sceneContainer:Create("PLScene::SceneContainer", "Serenael", "Scale='0.05 0.05 0.05'")
if serenaelContainer ~= nil then
	serenaelContainer:LoadByFilename("data/msh/serenael.x", "Quality=1")
end
\end{lstlisting}
in order to load in a scene and proving the loader with additional parameters, or even decide which concrete loader method should be used by writing
\begin{lstlisting}[language=lua]
local serenaelContainer = sceneContainer:Create("PLScene::SceneContainer", "Serenael", "Scale='0.05 0.05 0.05'")
if serenaelContainer ~= nil then
	serenaelContainer:LoadByFilename("data/msh/serenael.x", "Quality=1", "LoadParams")
end
\end{lstlisting}
Due to the flexibility the \ac{RTTI} introduces, the available loader parameters depend on the used loader implementation. If there's no need for additional settings,
\begin{lstlisting}[language=lua]
local serenaelContainer = sceneContainer:Create("PLScene::SceneContainer", "Serenael", "Scale='0.05 0.05 0.05'")
if serenaelContainer ~= nil then
	serenaelContainer:LoadByFilename("data/msh/serenael.x")
end
\end{lstlisting}
or
\begin{lstlisting}[language=lua]
sceneContainer:Create("PLScene::SceneContainer", "Serenael", "Scale='0.05 0.05 0.05' Filename='data/msh/serenael.x'")
\end{lstlisting}
...  also does the job.
