\chapter{Virtual Standard Controller}
\label{Appendix:VirtualStandardController}
The \emph{virtual standard controller} is an abstraction of physical devices basing on actions, like \emph{jump}, instead of physical buttons, axis and so on. So, only the most common physical devices mouse and keyboard are directly mapped by the virtual standard controller. The rest of the available devices is mapped onto the offered standard controls within this controller.

In this appendix, the global variable \emph{this} points to the C++ \ac{RTTI} application class instance invoking the script. By writing \begin{quote}this:GetInputController()\end{quote} within a Lua script, one can request the instance of the input controller \ac{RTTI} class the application is using. Within the PixelLight application framework, this is an instance of the virtual standard controller by default. The information in this appendix assumes that the application, your script is written for, is using this default behaviour instead of implementing an individual input controller.


\paragraph{Mapped Mouse}
\begin{center}
	\centering
	\begin{longtable}{ | l | p{10cm} |}
	\hline
	Name	& Description\\ \hline
	MouseX			&	X axis\\
	MouseY			&	Y axis\\
	MouseWheel		&	Mouse wheel\\
	MouseLeft		&	Left mouse button (mouse button 0)\\
	MouseRight		&	Right mouse button (mouse button 1)\\
	MouseMiddle		&	Middle mouse button (mouse button 2)\\
	MouseButton4	&	Mouse button 4\\
	MouseButton5	&	Mouse button 5\\
	MouseButton6	&	Mouse button 6\\
	MouseButton7	&	Mouse button 7\\
	MouseButton8	&	Mouse button 8\\
	MouseButton9	&	Mouse button 9\\
	MouseButton10	&	Mouse button 10\\
	MouseButton11	&	Mouse button 11\\
	MouseButton12	& 	Mouse button 12\\
	\hline
	\end{longtable}
\end{center}


\paragraph{Mapped Keyboard}
\begin{center}
	\centering
	\begin{longtable}{ | l | p{10cm} |}
	\hline
	Name	& Description \\ \hline
	KeyboardBackspace		&	Backspace\\
	KeyboardTab				&	Tabulator\\
	KeyboardClear			&	Clear (not available everywhere)\\
	KeyboardReturn			&	Return (often the same as "Enter")\\
	KeyboardShift			&	Shift\\
	KeyboardControl			&	Control ("Ctrl")\\
	KeyboardAlt				&	Alt\\
	KeyboardPause			&	Pause\\
	KeyboardCapsLock		&	Caps lock\\
	KeyboardEscape			&	Escape\\
	KeyboardSpace			&	Space\\
	KeyboardPageUp			&	Page up\\
	KeyboardPageDown		&	Page down\\
	KeyboardEnd				&	End\\
	KeyboardHome			&	Home\\
	KeyboardLeft			&	Left arrow\\
	KeyboardUp				&	Up arrow\\
	KeyboardRight			&	Right arrow\\
	KeyboardDown			&	Down arrow\\
	KeyboardSelect			&	Select (not available everywhere)\\
	KeyboardExecute			&	Execute (not available everywhere)\\
	KeyboardPrint			&	Print screen\\
	KeyboardInsert			&	Insert\\
	KeyboardDelete			&	Delete\\
	KeyboardHelp			&	Help (not available everywhere)\\
	Keyboard0				&	0\\
	Keyboard1				&	1\\
	Keyboard2				&	2\\
	Keyboard3				&	3\\
	Keyboard4				&	4\\
	Keyboard5				&	5\\
	Keyboard6				&	6\\
	Keyboard7				&	7\\
	Keyboard8				&	8\\
	Keyboard9				&	9\\
	KeyboardA				&	A\\
	KeyboardB				&	B\\
	KeyboardC				&	C\\
	KeyboardD				&	D\\
	KeyboardE				&	E\\
	KeyboardF				&	F\\
	KeyboardG				&	G\\
	KeyboardH				&	H\\
	KeyboardI				&	I\\
	KeyboardJ				&	J\\
	KeyboardK				&	K\\
	KeyboardL				&	L\\
	KeyboardM				&	M\\
	KeyboardN				&	N\\
	KeyboardO				&	O\\
	KeyboardP				&	P\\
	KeyboardQ				&	Q\\
	KeyboardR				&	R\\
	KeyboardS				&	S\\
	KeyboardT				&	T\\
	KeyboardU				&	U\\
	KeyboardV				&	V\\
	KeyboardW				&	W\\
	KeyboardX				&	X\\
	KeyboardY				&	Y\\
	KeyboardZ				&	Z\\
	KeyboardNumpad0			&	Numpad 0\\
	KeyboardNumpad1			&	Numpad 1\\
	KeyboardNumpad2			&	Numpad 2\\
	KeyboardNumpad3			&	Numpad 3\\
	KeyboardNumpad4			&	Numpad 4\\
	KeyboardNumpad5			&	Numpad 5\\
	KeyboardNumpad6			&	Numpad 6\\
	KeyboardNumpad7			&	Numpad 7\\
	KeyboardNumpad8			&	Numpad 8\\
	KeyboardNumpad9			&	Numpad 9\\
	KeyboardNumpadMultiply	&	Numpad Multiply\\
	KeyboardNumpadAdd		&	Numpad Add\\
	KeyboardNumpadSeparator	&	Numpad Separator\\
	KeyboardNumpadSubtract	&	Numpad Subtract\\
	KeyboardNumpadDecimal	&	Numpad Decimal\\
	KeyboardNumpadDivide	&	Numpad Divide\\
	KeyboardF1				&	F1\\
	KeyboardF2				&	F2\\
	KeyboardF3				&	F3\\
	KeyboardF4				&	F4\\
	KeyboardF5				&	F5\\
	KeyboardF6				&	F6\\
	KeyboardF7				&	F7\\
	KeyboardF8				&	F8\\
	KeyboardF9				&	F9\\
	KeyboardF10				&	F10\\
	KeyboardF11				&	F11\\
	KeyboardF12				&	F12\\
	KeyboardNumLock			&	NumLock\\
	KeyboardScrollLock		&	ScrollLock\\
	KeyboardCircumflex		&	Circumflex\\
	\hline
	\end{longtable}
\end{center}


\paragraph{Main Character Controls}
\begin{center}
	\centering
	\begin{longtable}{ | l | p{10cm} |}
	\hline
	Name	& Description \\ \hline
	TransX		&	X translation axis: Strafe left/right (+/-)\\
	TransY		&	Y translation axis: Move up/down (+/-)\\
	TransZ		&	Z translation axis: Move forwards/backwards (+/-)\\
	Pan			&	Keep pressed to pan\\
	PanX		&	X pan translation axis: Strafe left/right (+/-)\\
	PanY		&	Y pan translation axis: Move up/down (+/-)\\
	PanZ		&	Z pan translation axis: Move forwards/backwards (+/-)\\
	RotX		&	X rotation axis: Pitch (also called 'bank') change is moving the nose down and the tail up (or vice-versa)\\
	RotY		&	Y rotation axis: Yaw (also called 'heading') change is turning to the left or right\\
	RotZ		&	Z rotation axis: Roll (also called 'attitude') change is moving one wingtip up and the other down\\
	Rotate		&	Keep pressed to rotate\\
	Forward		&	Move forwards\\
	Backward	&	Move backwards\\
	Left		&	Move (rotate) left\\
	Right		&	Move (rotate) right\\
	StrafeLeft	&	Strafe left\\
	StrafeRight	&	Strafe right\\
	Up			&	Move up\\
	Down		&	Move down\\
	Run			&	Keep pressed to run\\
	Crouch		&	Keep pressed to crouch\\
	Jump		&	Jump\\
	Zoom		&	Keep pressed to zoom\\
	ZoomAxis	&	Zoom axis to zoom in or out (+/-)\\
	Button1		&	Button for action 1\\
	Button2		&	Button for action 2\\
	Button3		&	Button for action 3\\
	Button4		&	Button for action 4\\
	Button5		&	Button for action 5\\
	\hline
	\end{longtable}
\end{center}


\paragraph{Interaction}
\begin{center}
	\centering
	\begin{longtable}{ | l | p{10cm} |}
	\hline
	Name	& Description \\ \hline
	Pickup			&	Keep pressed to pickup\\
	Throw			&	Throw the picked object\\
	IncreaseForce	&	Keep pressed to increase the force applied to the picked object\\
	DecreaseForce	&	Keep pressed to decrease the force applied to the picked object\\
	PushPull		&	Used to push/pull the picked object\\
	\hline
	\end{longtable}
\end{center}
