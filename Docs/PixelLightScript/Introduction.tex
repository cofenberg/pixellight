\chapter{Introduction}


\paragraph{Target Audience}
This document is meant for programmers and artists capable of doing some scripting.


\paragraph{Motivation}
This script documentation is intended for script programmers. While this document also talks about some quite basic stuff, the targeted audience has already some fundamental programming experience. If you're totally new to the field of programming in general, it's highly recommended to read some beginner literature first \footnote{For example \emph{Programming in Lua (first edition)} which is online available at \url{http://www.lua.org/pil/} if you're just interested in scripting using Lua}.

The reason for creating the script interface was to have a minimalistic and universal interface to work with script languages. Instead of inventing an own script language, we use a backend design pattern to use already available common script languages like \emph{Lua}, \emph{JavaScript}, \emph{Python} or \emph{AngelScript}.

Why should you care about scripting in the first place when PixelLight is written in C++ and you have the whole C++ \ac{API} and even the complete source code of the project at our hands? Well, let me say it this way: Not every programmer is a programmer. What's meant by this statement is, that within development teams not every team member is able to or want's to work in C++. When it comes to implement application logic like \begin{quote}''When the user clicks on this door then let an anvil fall down on him''\end{quote}, there's no real need to let everything be written by experienced (and therefore usually expensive) software developers within C++. Let's say a graphics artist has finished his work and has enough of colourful things for a few hours - why shouldn't he be allowed to implement some simple application logic? By using scripting, it's usually no big deal to let other persons than experienced software developers do some coding.




\section{Interactive Realtime \ac{RTTI} Browsing \ac{GUI}}
The script system heavily relies on the PixelLight \ac{RTTI} system. So, you probably may want to know which classes, attributes and so on are available for usage. Please note that the purpose of this document is not to describe each and every possible global function, \ac{RTTI} classes, methods, attributes and so on. Due to the highly plugin driven architecture of PixelLight, this would be impossible anyway. There are several chapters about fundamental \ac{RTTI} classes, but that's just a tiny portion of what's available. For the rest, please refer to the interactive realtime \ac{RTTI} browsing \ac{GUI} which can be accessed by using for instance the tool \emph{PLViewerQt} which comes with the \ac{SDK}. This way, the \emph{documentation} is automatically always up-to-date and also contains your own, custom extensions.




\section{External Dependences}
There's no individual \emph{PLScript} project, it's part of \emph{PLCore}. The \textbf{PLCore} library only depends on \emph{zlib} and \emph{libpcre}, both are statically linked in.
