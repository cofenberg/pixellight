\chapter{Introduction}


\paragraph{Motivation}
The PixelLight math component provides common required tools like vector and matrix classes and therefore implements features, which are especially useful when dealing this 3D graphics.




\section{External Dependences}
PLMath only depends on the \textbf{PLCore} library and doesn't use any external third party libraries.




\section{Transforming 2D to 3D and reversed}


\paragraph{Overview}
Sometimes it's required to project a 3D world position onto the screen to get the 2D x/y display position of it or you want to find out on which 3D world position a 2D position is on. This can be seen as the interface between the flat 2D screen and the 3D world.


\paragraph{3D to 2D}
\emph{Vector3::To2DCoordinate()} returns the 2D screen coordinate corresponding to a given 3D world coordinate using the given camera settings. This can be useful if you e.g. want to set the mouse cursor over a given 3D object.


\paragraph{2D to 3D}
\emph{Vector3::To3DCoordinate()} returns the 3D coordinate corresponding to a given 2D screen coordinate using the given camera settings. 2D to 3D transformation is extremely useful if you want to perform picking were the user is able to select a 3D object by using the mouse. Because picking is a collision task were we have to find the intersection point of the line (camera position to mouse position) and the meshes, this is not handled in this document. Have a look at the \emph{PixelLight API documentation} for more information how to perform picking.





\section{Additional tool functions}
\emph{PlaneSet::CreateSelectionPlanes()} takes a 2D start and end point and will return a plane set with the \emph{selection planes}. This plane set can be seen as \emph{selection frustum} and you can test your objects against it to see what's selected and what's not.
