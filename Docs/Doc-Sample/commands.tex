%----- Section: Commands -------------------------
\section{Commands}


%----- SubSection: Timestamp ---------------------
\subsection{Timestamp}


%----- SubSubSection: Demonstration --------------
\subsubsection{Demonstration}
You can use the commands \verb#\printdate#, \verb#\printtime#
and \verb#\timestamp# to print the current date and time:\\
\\
\begin{tabular}{ll}
Date: & \printdate\\
Time: & \printtime\\
Timestamp: & \timestamp\\
\end{tabular}


%----- SubSubSection: Source ---------------------
\subsubsection{Source}
Source for the demonstration:

\begin{verbatim}
  You can use the commands \verb#\printdate#, \verb#\printtime#
  and \verb#\timestamp# to print the current date and time:\\
  \\
  \begin{tabular}{ll}
  Date: & \printdate\\
  Time: & \printtime\\
  Timestamp: & \timestamp\\
  \end{tabular}
\end{verbatim}
\newpage


%----- SubSection: Boxes -------------------------
\subsection{Boxes}


%----- SubSubSection: Demonstration --------------
\subsubsection{Demonstration}
Here are some examples of using the new box commands defined in tools.sty:\\

\noindent
\textbf{advbox:}\\

\begin{advbox}
This is a simple framed box produced by the \textbf{advbox} environment,
that can be used to frame a text.\\
Multiple lines and paragraphs can be typed inside the box, as well as
\begin{Verbatim}
        some      text
            verbatim
\end{Verbatim}
\end{advbox}\\
\\

\noindent
\textbf{cadvbox:}\\

\definecolor{mygray}{rgb}{0.9,0.9,0.9}
\begin{cadvbox}{black}{mygray}
This is a colored box, it is produced by the \textbf{cadvbox} command.
The first argument is the frame color, the second argument is the
background color. Everything else works exactly like \textbf{advbox}.
\end{cadvbox}


%----- SubSubSection: Source ---------------------
\subsubsection{Source}
Source for the demonstration:

\begin{verbatim}
\begin{advbox}
This is a simple framed box produced by the \textbf{advbox} environment,
that can be used to frame a text.\\
Multiple lines and paragraphs can be typed inside the box, as well as
\ begin{verbatim}
        some      text
            verbatim
\ end{verbatim}
\end{advbox}\\

\definecolor{mygray}{rgb}{0.9,0.9,0.9}
\begin{cadvbox}{black}{mygray}
This is a colored box, it is produced by the \textbf{cadvbox} command.
The first argument is the frame color, the second argument is the
background color. Everything else works exactly like \textbf{advbox}.
\end{cadvbox}
\end{verbatim}
\newpage


%----- SubSection: Environments ------------------
\subsection{Environments}


%----- SubSubSection: Demonstration --------------
\subsubsection{Demonstration}
There are two new environments for defining often used sections (notes and code) in a document:\\

\noindent
\textbf{NOTE:}\\

\begin{NOTE}
This is a note for the reader. Use the \textbf{CODE} environment for this,
as an optional parameter you can add a simple description that will be
printed in the headline.
\end{NOTE}
\vspace{0.4cm}

\noindent
\textbf{CODE:}\\

\begin{CODE}[Some pseudo code]
  if a=b
    proc1;
  else
    proc2;
  fi
\end{CODE}


%----- SubSubSection: Source ---------------------
\subsubsection{Source}
\begin{verbatim}
\begin{NOTE}
This is a note for the reader. Use the \textbf{CODE} environment for this,
as an optional parameter you can add a simple description that will be
printed in the headline.
\end{NOTE}

\begin{CODE}[Some pseudo code]
  if a=b
    proc1;
  else
    proc2;
  fi
\end{CODE}
\end{verbatim}
\newpage


%----- SubSection: Links within the document -----
\subsection{Links within the document}


%----- SubSubSection: Demonstration --------------
\hypertarget{link1}{}
\subsubsection{Demonstration}
Here you can jump to a \hyperlink{link2}{target} by clicking on the link.


%----- SubSubSection: Source ---------------------
\subsubsection{Source}
Source for the demonstration:

\begin{verbatim}
  \hypertarget{link1}{}
  \subsection{Demonstration}
  Here you can jump to a \hyperlink{link2}{target} by clicking on the link.
\end{verbatim}

\noindent
Source for the link target:
\begin{verbatim}
  \hypertarget{link2}{}
  \subsection{Link Target}
  This is just a target to demonstrate the link command...
  Go \hyperlink{link1}{back} now
\end{verbatim}
\newpage


%----- SubSection: Hyperlinks --------------------
\subsection{Hyperlinks}


%----- SubSubSection: Demonstration --------------
\subsubsection{Demonstration}
There are two ways to jump to a URL:
\begin{itemize}
\item Jump to this \href{http://www.pixellight.org}{website}
\item Jump to \url{http://www.pixellight.org}
\end{itemize}


%----- SubSubSection: Source ---------------------
\subsubsection{Source}
Source for the demonstration:

\begin{verbatim}
  There are two ways to jump to a URL:
  \begin{itemize}
  \item Jump to this \href{http://www.pixellight.org}{website}
  \item Jump to \url{http://www.pixellight.org}
  \end{itemize}
\end{verbatim}
\newpage
